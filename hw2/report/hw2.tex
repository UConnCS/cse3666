\documentclass{article}
\usepackage[utf8]{inputenc}

\title{CSE3666 — Homework 2}
\author{Mike Medved}
\date{February 11th, 2022}

\usepackage{color}
\usepackage{amsthm}
\usepackage{amssymb} 
\usepackage{amsmath}
\usepackage[margin=1in]{geometry} 
\usepackage{listings}
\usepackage{xcolor}
\usepackage{listings}

\lstset{frame=tb,
    language=[LaTeX]TeX,
    aboveskip=3mm,
    belowskip=3mm,
    showstringspaces=false,
    columns=flexible,
    basicstyle={\small\ttfamily},
    numbers=none,
    numberstyle=\tiny\color{gray},
    keywordstyle=\color{blue},
    commentstyle=\color{dkgreen},
    stringstyle=\color{mauve},
    breaklines=true,
    breakatwhitespace=true,
    tabsize=3
}

\lstdefinelanguage[mips]{Assembler}{%
  % so listings can detect directives and register names
  alsoletter={.\$},
  % strings, characters, and comments
  morestring=[b]",
  morestring=[b]',
  morecomment=[l]\#,
  % instructions
  morekeywords={[1]abs,abs.d,abs.s,add,add.d,add.s,addi,addiu,addu,%
    and,andi,b,bc1f,bc1t,beq,beqz,bge,bgeu,bgez,bgezal,bgt,bgtu,%
    bgtz,ble,bleu,blez,blt,bltu,bltz,bltzal,bne,bnez,break,c.eq.d,%
    c.eq.s,c.le.d,c.le.s,c.lt.d,c.lt.s,ceil.w.d,ceil.w.s,clo,clz,%
    cvt.d.s,cvt.d.w,cvt.s.d,cvt.s.w,cvt.w.d,cvt.w.s,div,div.d,div.s,%
    divu,ecall,eret,floor.w.d,floor.w.s,j,jal,jalr,jr,l.d,l.s,la,lb,lbu,%
    ld,ldc1,lh,lhu,li,ll,lui,lw,lwc1,lwl,lwr,madd,maddu,mfc0,mfc1,%
    mfc1.d,mfhi,mflo,mov.d,mov.s,move,movf,movf.d,movf.s,movn,movn.d,%
    movn.s,movt,movt.d,movt.s,movz,movz.d,movz.s,msub,msubu,mtc0,mtc1,%
    mtc1.d,mthi,mtlo,mul,mul.d,mul.s,mulo,mulou,mult,multu,mulu,mv,neg,%
    neg.d,neg.s,negu,nop,nor,not,or,ori,rem,remu,rol,ror,round.w.d,%
    round.w.s,s.d,s.s,sb,sc,sd,sdc1,seq,sge,sgeu,sgt,sgtu,sh,sle,%
    sleu,sll,sllv,slt,slti,sltiu,sltu,sne,sqrt.d,sqrt.s,sra,srav,srl,%
    srlv,sub,sub.d,sub.s,subi,subiu,subu,sw,swc1,swl,swr,syscall,teq,%
    teqi,tge,tgei,tgeiu,tgeu,tlt,tlti,tltiu,tltu,tne,tnei,trunc.w.d,%
    trunc.w.s,ulh,ulhu,ulw,ush,usw,xor,xori},
  % assembler directives
  morekeywords={[2].align,.ascii,.asciiz,.byte,.data,.double,.extern,%
    .float,.globl,.half,.kdata,.ktext,.set,.space,.text,.word},
  % register names
  morekeywords={[3]\$0,\$1,\$2,\$3,\$4,\$5,\$6,\$7,\$8,\$9,\$10,\$11,%
    \$12,\$13,\$14,\$15,\$16,\$17,\$18,\$19,\$20,\$21,\$22,\$23,\$24,%
    \$25,\$26,\$27,\$28,\$29,\$30,\$31,%
    \$zero,\$at,\$v0,\$v1,\$a0,\$a1,\$a2,\$a3,\$t0,\$t1,\$t2,\$t3,\$t4,
    \$t5,\$t6,\$t7,\$s0,\$s1,\$s2,\$s3,\$s4,\$s5,\$s6,\$s7,\$t8,\$t9,%
    \$k0,\$k1,\$gp,\$sp,\$fp,\$ra},
}[strings,comments,keywords]

\definecolor{CommentGreen}{rgb}{0,.6,0}
\lstset{
  language=[mips]Assembler,
  escapechar=@, % include LaTeX code between `@' characters
  keepspaces,   % needed to preserve spacing with lstinline
  basicstyle=\small\ttfamily\bfseries,
  commentstyle=\color{CommentGreen},
  stringstyle=\color{cyan},
  showstringspaces=false,
  keywordstyle=[1]\color{blue},    % instructions
  keywordstyle=[2]\color{magenta}, % directives
  keywordstyle=[3]\color{red},     % registers
}

\begin{document}

\maketitle

\section{Prompt}
Loop unrolling is an optimization technique to improve the performance of programs. In
the second implementation, we unroll the loop and process four array elements in A in
each iteration. The unrolled loop in C is shown below. Translate the loop to RISC-V
instructions. Try to minimize the number of instructions that are executed. Explain your
code. How many instructions will be executed for the new loop?

\hfill
\begin{lstlisting}[language=C]
for (i = 0; i < 100; i += 4) {
    B[i] = A[i] + 4;
    B[i+1] = A[i+1] + 4;
    B[i+2] = A[i+2] + 4;
    B[i+3] = A[i+3] + 4;
}
\end{lstlisting}

\section{Deliverables}
    \begin{lstlisting}
    # CSE 3666 Homework 2 - Question 1b
    .globl main
    .text
    
    main:
        li s1, 0           # i = 0
        li s4, 100         # i < 100
        beq x0, x0, test   # we know s1 < s4
    
    loop:
        slli t0, s1, 2 # t0 = i * 4
        
        # B[i] = A[i] + 4
        add t2, t0, s2 # compute addr of A[i]
        lw t1, 0(t2)   # retrieve and store A[i] in t1 register
        addi t1, t1, 4 # A[i] + 4
        add t3, t0, s3 # compute addr of B[i]
        sw t1, 0(t3)   # store value of t1 (A[i] + 4) into B[i]
        
        # B[i+1] = A[i+1] + 4
        # since we already have the address of A[i], 
        # load the address+4 which corresponds to A[i+1]
        # since ints are 4-bytes.
        lw t1, 4(t2)
        addi t1, t1, 4 # A[i+1] + 4
        sw t1, 4(t3)   # store value of t1 (A[i+1] + 4) into B[i+1]
        
        # B[i+2] = A[i+2] + 4
        # knowing the address of A[i], add an offset of 8
        # to get the second element of the word array.
        lw t1, 8(t2) 
        addi t1, t1, 4 # A[i+2] + 4
        sw t1, 8(t3)   # store value of t1 (A[i+2] + 4) into B[i+2]
        
        # B[i+3] = A[i+3] + 4
        # again, knowing the base address, add an offset of 12
        # to get the third element of the word array.
        lw t1, 12(t2)
        addi t1, t1, 4 # A[i+3] + 4
        sw 51, 12(t3)  # store value of t1 (A[i+3] + 4) into B[i+3]
        
        addi s1, s1, 4 # i += 4
    
    test:
        bne s1, s4, loop
    \end{lstlisting}

\end{document}
