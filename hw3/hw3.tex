\documentclass{article}
\usepackage[utf8]{inputenc}

\title{CSE3666 — Homework 3}
\author{Mike Medved}
\date{February 21st, 2022}

\usepackage{color}
\usepackage{amsthm}
\usepackage{amssymb} 
\usepackage{amsmath}
\usepackage[margin=1in]{geometry} 
\usepackage{listings}
\usepackage{xcolor}
\usepackage{listings}

\lstset{frame=tb,
    language=[LaTeX]TeX,
    aboveskip=3mm,
    belowskip=3mm,
    showstringspaces=false,
    columns=flexible,
    basicstyle={\small\ttfamily},
    numbers=none,
    numberstyle=\tiny\color{gray},
    keywordstyle=\color{blue},
    commentstyle=\color{dkgreen},
    stringstyle=\color{mauve},
    breaklines=true,
    breakatwhitespace=true,
    tabsize=3
}

\lstdefinelanguage[mips]{Assembler}{%
  % so listings can detect directives and register names
  alsoletter={.\$},
  % strings, characters, and comments
  morestring=[b]",
  morestring=[b]',
  morecomment=[l]\#,
  % instructions
  morekeywords={[1]abs,abs.d,abs.s,add,add.d,add.s,addi,addiu,addu,%
    and,andi,b,bc1f,bc1t,beq,beqz,bge,bgeu,bgez,bgezal,bgt,bgtu,%
    bgtz,ble,bleu,blez,blt,bltu,bltz,bltzal,bne,bnez,break,c.eq.d,%
    c.eq.s,c.le.d,c.le.s,c.lt.d,c.lt.s,ceil.w.d,ceil.w.s,clo,clz,%
    cvt.d.s,cvt.d.w,cvt.s.d,cvt.s.w,cvt.w.d,cvt.w.s,div,div.d,div.s,%
    divu,ecall,eret,floor.w.d,floor.w.s,j,jal,jalr,jr,l.d,l.s,la,lb,lbu,%
    ld,ldc1,lh,lhu,li,ll,lui,lw,lwc1,lwl,lwr,madd,maddu,mfc0,mfc1,%
    mfc1.d,mfhi,mflo,mov.d,mov.s,move,movf,movf.d,movf.s,movn,movn.d,%
    movn.s,movt,movt.d,movt.s,movz,movz.d,movz.s,msub,msubu,mtc0,mtc1,%
    mtc1.d,mthi,mtlo,mul,mul.d,mul.s,mulo,mulou,mult,multu,mulu,mv,neg,%
    neg.d,neg.s,negu,nop,nor,not,or,ori,rem,remu,rol,ror,round.w.d,%
    round.w.s,s.d,s.s,sb,sc,sd,sdc1,seq,sge,sgeu,sgt,sgtu,sh,sle,%
    sleu,sll,slli,sllv,slt,slti,sltiu,sltu,sne,sqrt.d,sqrt.s,sra,srai,srav,srl,%
    srlv,sub,sub.d,sub.s,subi,subiu,subu,sw,swc1,swl,swr,syscall,teq,%
    teqi,tge,tgei,tgeiu,tgeu,tlt,tlti,tltiu,tltu,tne,tnei,trunc.w.d,%
    trunc.w.s,ulh,ulhu,ulw,ush,usw,xor,xori},
  % assembler directives
  morekeywords={[2].align,.ascii,.asciiz,.byte,.data,.double,.extern,%
    .float,.globl,.half,.kdata,.ktext,.set,.space,.text,.word},
  % register names
  morekeywords={[3]\$0,\$1,\$2,\$3,\$4,\$5,\$6,\$7,\$8,\$9,\$10,\$11,%
    \$12,\$13,\$14,\$15,\$16,\$17,\$18,\$19,\$20,\$21,\$22,\$23,\$24,%
    \$25,\$26,\$27,\$28,\$29,\$30,\$31,%
    \$zero,\$at,\$v0,\$v1,\$a0,\$a1,\$a2,\$a3,\$t0,\$t1,\$t2,\$t3,\$t4,
    \$t5,\$t6,\$t7,\$s0,\$s1,\$s2,\$s3,\$s4,\$s5,\$s6,\$s7,\$t8,\$t9,%
    \$k0,\$k1,\$gp,\$sp,\$fp,\$ra},
}[strings,comments,keywords]

\definecolor{CommentGreen}{rgb}{0,.6,0}
\lstset{
  language=[mips]Assembler,
  escapechar=@, % include LaTeX code between `@' characters
  keepspaces,   % needed to preserve spacing with lstinline
  basicstyle=\small\ttfamily\bfseries,
  commentstyle=\color{CommentGreen},
  stringstyle=\color{cyan},
  showstringspaces=false,
  keywordstyle=[1]\color{blue},    % instructions
  keywordstyle=[2]\color{magenta}, % directives
  keywordstyle=[3]\color{red},     % registers
}

\begin{document}

\maketitle

\section{Question 5}
Translate function $f$ in the following C code to RISC-V assembly code. Assume function $g$ has already been implemented. The constraints are:
\begin{enumerate}
    \item Allocate register s1 to sum, and register s2 to i.
    \item Save registers at the beginning of the function and restore them before the exit.
    \item There are no load or store instructions in the loop. If we want to preserve values across function calls, place the value in a saved register before the loop. For example, we keep variable i in register s2.
\end{enumerate}

\hfill
\begin{lstlisting}[language=C]
    // prototype of g
    // the first argument of g is an address of an integer
    int g(int * a, int i);

    int f(int d[1024]) {
        int sum = 0;
        for (int i = 0; i < 1024; i += 1) {
            sum += g(&d[i], i); // pass d[i]’s address to g
        }

        return sum;
    }
\end{lstlisting}

\hfill \break
Your code should follow the flow of the C code. Write concise comments. Clearly mark
instructions for saving registers, loop control, function, restoring register, and so on.

\hfill \break
Reminder: the callee can change any temporary and argument registers.

\break
\section{Deliverable}
    \begin{lstlisting}
    # CSE 3666 Homework 3 - Question 5
    f:
        # allocate 20 bytes on the stack
        # for storing ra, s1, s2, s3, and s4.
        addi sp, sp, -20

        # store ra, s1, s2, s3, and s4 respectively
        sw   ra, 16(sp)
        sw   s4, 12(sp)
        sw   s3, 8(sp)
        sw   s2, 4(sp)
        sw   s1, 0(sp)

        addi s1, x0, 0    # initialize sum as 0
        addi s2, x0, 0    # initialize loop counter (i) as 0
        addi s3, x0, 1024 # save the loop bound (1024)
        addi s4, a0, 0    # save the base address of d

    loop:
        slli a0, s2, 2    # compute offset for current index, i *= 4
        add  a0, a0, s4   # save the address of d[i] in a0 for passing to g
        addi a1, s2, 0    # save the loop counter in a1 for passing to g
        jal  ra, g        # invoke g(&d[i], i)

        # with a0 as the result of g,
        # add to the current sum and continue
        add  s1, s1, a0
        
        addi s2, s2, 1    # increment the loop counter, i++
        blt  s2, s3, loop # if i < 1024, rerun the loop
        addi a0, s1, 0    # save the sum in a0
        
        # restore ra, s1, s2, s3, and s4
        lw   ra, 16(sp)
        lw   s4, 12(sp)
        lw   s3, 8(sp)
        lw   s2, 4(sp)
        lw   s1, 0(sp)

        # move stack pointer back to start
        addi sp, sp, 20

        # return from current context
        jr ra
    \end{lstlisting}

\break
\section{Question 6}
Translate function msort() in the following C code to RISC-V assembly code. Assume merge() and copy() are already implemented. The array passed to msort() has at most 256 elements.

\hfill \break
Your code should follow the flow of the C code. Write concise comments. Clearly mark instructions for saving registers, function calls, restoring register, and so on.

\hfill \break
To make the code easier to read, we can change sp twice at the beginning of the function: once for saving registers and once for allocating memory for array c.

\hfill \break
The function should have only one exit. There is only one return instruction.

\hfill \break
Another reminder: the callees can change any temporary and argument registers.

\begin{lstlisting}[language=C]
    void merge(int c[], int d1[], int n1, int d2[], int n2);
    void copy(int d[], int c[], int n);
    
    void msort(int d[], int n) {
        int c[256];
        if (n <= 1)
            return;

        int n1 = n / 2;
        msort(d, n1);
        msort(&d[n1], n – n1); // &d[n1] means the address of d[n1]
        merge(c, d, n1, &d[n1], n – n1);
        copy(d, c, n);
    }
\end{lstlisting}

\break
\section{Deliverable}
    \begin{lstlisting}
        # CSE 3666 Homework 3 - Question 6
        msort:
            # save ra, s1, s2
            # allocate 1036 bytes on the stack
            # to be able to store up to 256 elements
            addi sp, sp, -1036

            # store ra, s1, and s2 respectively
            # with s1 being the given array d
            # growing downwards on the stack
            # as elements are populated
            sw   ra, 1032(sp)
            sw   s2, 1028(sp)
            sw   s1, 1024(sp) 

            addi s1, a0, 0    # save the given array in s1
            addi s2, a1, 0    # save the array partition in s2
            addi t0, t0, 2    # n <= 1?

            blt  a1, t0, exit # if n <= 1, exit
            addi t0, t0, 257  # store 257 in t0 to use for the n > 256 check
            bge  a1, t0, exit # if n > 256, exit

            srai a1, s2, 1    # compute n1 = n / 2 and store in a1
            jal  ra, msort    # invoke msort(d, n1)

            srai a1, s2, 1    # compute n = n / 2
            slli a0, a1, 2    # compute n1 = n – n1 and store in a0
            add  a0, a0, s1   # compute the address of d[n1]
            jal  ra, msort    # invoke msort(&d[n1], n - n1)
            
            addi a0, sp, 0    # save the address of c in a0
            addi a1, s1, 0    # save the address of d in a1
            srai a2, s2, 1    # compute n1 = n / 2 and store in a2
            slli a3, a2, 2    # compute n2 = n – n1 and store in a3
            add  a3, a3, s1   # compute the address of d[n1]
            sub  a4, s2, a2   # compute n2 = n – n1 again
            jal  ra, merge    # invoke merge(c, d, n1, &d[n1], n – n1)
            
            addi a0, s1, 0    # save the address of d in a0
            addi a1, sp, 0    # save the address of c in a1
            addi a2, s2, 0    # save the array partition in a2
            jal ra, copy      # invoke copy(d, c, n)

        exit:
            # restore ra, s1, and s2
            lw ra, 1032(sp)
            lw s2, 1028(sp)
            lw s1, 1024(sp)

            # move stack pointer back to start
            addi sp, sp, 1036

            # return from current context
            jr ra
    \end{lstlisting}
\end{document}
