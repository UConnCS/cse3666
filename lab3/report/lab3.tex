\documentclass{article}
\usepackage[utf8]{inputenc}

\title{CSE3666 — Lab 3}
\author{Mike Medved}
\date{February 7th, 2022}

\usepackage{color}
\usepackage{amsthm}
\usepackage{amssymb} 
\usepackage{amsmath}
\usepackage[margin=1in]{geometry} 
\usepackage{listings}
\usepackage{xcolor}
\usepackage{listings}

\newcommand*\BitAnd{\mathbin{\&}}
\newcommand*\BitOr{\mathbin{|}}
\newcommand*\ShiftLeft{\ll}
\newcommand*\ShiftRight{\gg}
\newcommand*\BitNeg{\ensuremath{\mathord{\sim}}}

\lstset{frame=tb,
    language=[LaTeX]TeX,
    aboveskip=3mm,
    belowskip=3mm,
    showstringspaces=false,
    columns=flexible,
    basicstyle={\small\ttfamily},
    numbers=none,
    numberstyle=\tiny\color{gray},
    keywordstyle=\color{blue},
    commentstyle=\color{dkgreen},
    stringstyle=\color{mauve},
    breaklines=true,
    breakatwhitespace=true,
    tabsize=3
}

\lstdefinelanguage[mips]{Assembler}{%
  % so listings can detect directives and register names
  alsoletter={.\$},
  % strings, characters, and comments
  morestring=[b]",
  morestring=[b]',
  morecomment=[l]\#,
  % instructions
  morekeywords={[1]abs,abs.d,abs.s,add,add.d,add.s,addi,addiu,addu,%
    and,andi,b,bc1f,bc1t,beq,beqz,bge,bgeu,bgez,bgezal,bgt,bgtu,%
    bgtz,ble,bleu,blez,blt,bltu,bltz,bltzal,bne,bnez,break,c.eq.d,%
    c.eq.s,c.le.d,c.le.s,c.lt.d,c.lt.s,ceil.w.d,ceil.w.s,clo,clz,%
    cvt.d.s,cvt.d.w,cvt.s.d,cvt.s.w,cvt.w.d,cvt.w.s,div,div.d,div.s,%
    divu,ecall,eret,floor.w.d,floor.w.s,j,jal,jalr,jr,l.d,l.s,la,lb,lbu,%
    ld,ldc1,lh,lhu,li,ll,lui,lw,lwc1,lwl,lwr,madd,maddu,mfc0,mfc1,%
    mfc1.d,mfhi,mflo,mov.d,mov.s,move,movf,movf.d,movf.s,movn,movn.d,%
    movn.s,movt,movt.d,movt.s,movz,movz.d,movz.s,msub,msubu,mtc0,mtc1,%
    mtc1.d,mthi,mtlo,mul,mul.d,mul.s,mulo,mulou,mult,multu,mulu,mv,neg,%
    neg.d,neg.s,negu,nop,nor,not,or,ori,rem,remu,rol,ror,round.w.d,%
    round.w.s,s.d,s.s,sb,sc,sd,sdc1,seq,sge,sgeu,sgt,sgtu,sh,sle,%
    sleu,sll,sllv,slt,slti,sltiu,sltu,sne,sqrt.d,sqrt.s,sra,srav,srl,%
    srlv,sub,sub.d,sub.s,subi,subiu,subu,sw,swc1,swl,swr,syscall,teq,%
    teqi,tge,tgei,tgeiu,tgeu,tlt,tlti,tltiu,tltu,tne,tnei,trunc.w.d,%
    trunc.w.s,ulh,ulhu,ulw,ush,usw,xor,xori},
  % assembler directives
  morekeywords={[2].align,.ascii,.asciiz,.byte,.data,.double,.extern,%
    .float,.globl,.half,.kdata,.ktext,.set,.space,.text,.word},
  % register names
  morekeywords={[3]\$0,\$1,\$2,\$3,\$4,\$5,\$6,\$7,\$8,\$9,\$10,\$11,%
    \$12,\$13,\$14,\$15,\$16,\$17,\$18,\$19,\$20,\$21,\$22,\$23,\$24,%
    \$25,\$26,\$27,\$28,\$29,\$30,\$31,%
    \$zero,\$at,\$v0,\$v1,\$a0,\$a1,\$a2,\$a3,\$t0,\$t1,\$t2,\$t3,\$t4,
    \$t5,\$t6,\$t7,\$s0,\$s1,\$s2,\$s3,\$s4,\$s5,\$s6,\$s7,\$t8,\$t9,%
    \$k0,\$k1,\$gp,\$sp,\$fp,\$ra},
}[strings,comments,keywords]

\definecolor{CommentGreen}{rgb}{0,.6,0}
\lstset{
  language=[mips]Assembler,
  escapechar=@, % include LaTeX code between `@' characters
  keepspaces,   % needed to preserve spacing with lstinline
  basicstyle=\small\ttfamily\bfseries,
  commentstyle=\color{CommentGreen},
  stringstyle=\color{cyan},
  showstringspaces=false,
  keywordstyle=[1]\color{blue},    % instructions
  keywordstyle=[2]\color{magenta}, % directives
  keywordstyle=[3]\color{red},     % registers
}

\begin{document}

\maketitle

\section{Prompt}
In this lab, we write a program with RISC-V assembly language that performs
the following tasks:

\begin{enumerate}
    \item Read a string `str` from the console.
    \item Remove the spaces (ASCII value 32) in `str` and save the result in string `res`.
    \item Print $res$.
\end{enumerate} 

$\hfill \break$
Both strings are ASCII strings that end with a null character (NUL).

$\hfill \break$
Skeleton code is provided in `lab3.s`. Step 1 is already done. Study the code. Pseudocode for Step 2 is provided below. Step 3 is done by a system call.

$\hfill \break$
\textbf{Constraints in your code:} Use only argument registers (like `a0` and `a1`)
and temporary registers (like `t0` and `t1`). This will help you in the next
lab. There is no need to use pseudo instruction `la`. Addresses are already in
$a_0$ and $a_1$.

$\\$
Here are some example results:
\begin{lstlisting}[language=Java]
a b c
abc

RISC-V: The Free and Open RISC Instruction Set Architecture
RISC-V:TheFreeandOpenRISCInstructionSetArchitecture
\end{lstlisting}

$\hfill \break$
\textbf{Pseudocode}

\begin{lstlisting}[language=C]
    i = 0
    j = 0
    do
        c = str[i]
        if c != 32
           res[j] = c
           j += 1
        i += 1
    while c != 0
\end{lstlisting}

$\hfill \break$
\section{Deliverables}
    \begin{lstlisting}
    # CSE3666 Lab 3

    .data
    .align 2

    # allocating space for both strings
    str: .space  128
    res: .space  128

    .globl  main
    .text

    main:   
            # read a string into str
            # use pseudoinstruction la to load address into register
            la      a0, str
            li      a1, 100
            li      a7, 8
            ecall

            # a0 is the address of str, a1 is the address of res
            la      a1, res

            # initialize i, j to 0
            addi    t0, x0, 0 
            addi    t1, x0, 0 
            
            # initialize t5 to 32 (SPACE)
            addi 	t5, x0, 32
            j       loop
        
    loop:
            # compute address of str[i]:
            # given base address of str[0] stored in a0,
            # we can add i + a0 (since chars are 1-byte)
            # to get the resultant address of str[i]
            add     t2, t0, a0

            # load char from target address
            # str[0] + i, into register t3.
            lb      t3, 0(t2)

            addi    t0, t0, 1     # increment loop counter i
            bne     t3, t5, write # if str[i] != 32 (SPACE), enter routine to save char
            beq     t3, x0, exit  # if str[i] == 0  (NUL), enter exit routine
            beq     x0, x0, loop  # continue the loop if this point is reached

    write:
            # compute address of res[j]:
            # given base address of res[0] stored in a1,
            # we can add j + t1 (since chars are 1-byte)
            # to get resultant ddress of str[j]
            add     t4, t1, a1

            # load char from target address
            # res[0] + j, into register t3
            sb      t3, 0(t4)

            addi    t1, t1, 1     # increment loop counter j
            bne     t3, x0, loop  # if char != 0 (NUL), re-enter loop
            beq     t3, x0, exit  # if char == 0 (NUL), exit program

    exit:  
            # print resulting string stored in a1
            add     a0, x0, a1
            li      a7, 4
            ecall
    \end{lstlisting}

\hfill \break
\section{Run Examples}
\begin{lstlisting}[language=Java]
a b c
abc

abcd e
abcdef

ab c d e f gh i j
abcdefghij

abcdefghijklmnopqrstuvwxyz
abcdefghijklmnopqrstuvwxyz

This string has escape sequences \n and special characters **
Thisstringhasescapesequences\nandspecialcharacters**

RISC-V: The Free and Open RISC Instruction Set Architecture
RISC-V:TheFreeandOpenRISCInstructionSetArchitecture

This string has lots of spaces and other funny characters!!!
Thisstringhaslotsofspacesandotherfunnycharacters!!!
\end{lstlisting}

\section{Limitations}
This algorithm works for all inputted strings up to the specified 100-character maximum limit, including those with non-alphanumeric special characters, ASCII control sequences, and escape characters.

\end{document}
