\documentclass{article}
\usepackage[utf8]{inputenc}

\title{CSE3666 — Lab 1}
\author{Mike Medved}
\date{January 25th, 2022}

\usepackage{color}
\usepackage{amsthm}
\usepackage{amssymb} 
\usepackage{amsmath}
\usepackage[margin=1in]{geometry} 
\usepackage{listings}
\usepackage{xcolor}
\usepackage{listings}

\lstset{frame=tb,
    language=[LaTeX]TeX,
    aboveskip=3mm,
    belowskip=3mm,
    showstringspaces=false,
    columns=flexible,
    basicstyle={\small\ttfamily},
    numbers=none,
    numberstyle=\tiny\color{gray},
    keywordstyle=\color{blue},
    commentstyle=\color{dkgreen},
    stringstyle=\color{mauve},
    breaklines=true,
    breakatwhitespace=true,
    tabsize=3
}

\lstdefinelanguage[mips]{Assembler}{%
  % so listings can detect directives and register names
  alsoletter={.\$},
  % strings, characters, and comments
  morestring=[b]",
  morestring=[b]',
  morecomment=[l]\#,
  % instructions
  morekeywords={[1]abs,abs.d,abs.s,add,add.d,add.s,addi,addiu,addu,%
    and,andi,b,bc1f,bc1t,beq,beqz,bge,bgeu,bgez,bgezal,bgt,bgtu,%
    bgtz,ble,bleu,blez,blt,bltu,bltz,bltzal,bne,bnez,break,c.eq.d,%
    c.eq.s,c.le.d,c.le.s,c.lt.d,c.lt.s,ceil.w.d,ceil.w.s,clo,clz,%
    cvt.d.s,cvt.d.w,cvt.s.d,cvt.s.w,cvt.w.d,cvt.w.s,div,div.d,div.s,%
    divu,ecall,eret,floor.w.d,floor.w.s,j,jal,jalr,jr,l.d,l.s,la,lb,lbu,%
    ld,ldc1,lh,lhu,li,ll,lui,lw,lwc1,lwl,lwr,madd,maddu,mfc0,mfc1,%
    mfc1.d,mfhi,mflo,mov.d,mov.s,move,movf,movf.d,movf.s,movn,movn.d,%
    movn.s,movt,movt.d,movt.s,movz,movz.d,movz.s,msub,msubu,mtc0,mtc1,%
    mtc1.d,mthi,mtlo,mul,mul.d,mul.s,mulo,mulou,mult,multu,mulu,mv,neg,%
    neg.d,neg.s,negu,nop,nor,not,or,ori,rem,remu,rol,ror,round.w.d,%
    round.w.s,s.d,s.s,sb,sc,sd,sdc1,seq,sge,sgeu,sgt,sgtu,sh,sle,%
    sleu,sll,sllv,slt,slti,sltiu,sltu,sne,sqrt.d,sqrt.s,sra,srav,srl,%
    srlv,sub,sub.d,sub.s,subi,subiu,subu,sw,swc1,swl,swr,syscall,teq,%
    teqi,tge,tgei,tgeiu,tgeu,tlt,tlti,tltiu,tltu,tne,tnei,trunc.w.d,%
    trunc.w.s,ulh,ulhu,ulw,ush,usw,xor,xori},
  % assembler directives
  morekeywords={[2].align,.ascii,.asciiz,.byte,.data,.double,.extern,%
    .float,.globl,.half,.kdata,.ktext,.set,.space,.text,.word},
  % register names
  morekeywords={[3]\$0,\$1,\$2,\$3,\$4,\$5,\$6,\$7,\$8,\$9,\$10,\$11,%
    \$12,\$13,\$14,\$15,\$16,\$17,\$18,\$19,\$20,\$21,\$22,\$23,\$24,%
    \$25,\$26,\$27,\$28,\$29,\$30,\$31,%
    \$zero,\$at,\$v0,\$v1,\$a0,\$a1,\$a2,\$a3,\$t0,\$t1,\$t2,\$t3,\$t4,
    \$t5,\$t6,\$t7,\$s0,\$s1,\$s2,\$s3,\$s4,\$s5,\$s6,\$s7,\$t8,\$t9,%
    \$k0,\$k1,\$gp,\$sp,\$fp,\$ra},
}[strings,comments,keywords]

\definecolor{CommentGreen}{rgb}{0,.6,0}
\lstset{
  language=[mips]Assembler,
  escapechar=@, % include LaTeX code between `@' characters
  keepspaces,   % needed to preserve spacing with lstinline
  basicstyle=\small\ttfamily\bfseries,
  commentstyle=\color{CommentGreen},
  stringstyle=\color{cyan},
  showstringspaces=false,
  keywordstyle=[1]\color{blue},    % instructions
  keywordstyle=[2]\color{magenta}, % directives
  keywordstyle=[3]\color{red},     % registers
}

\begin{document}

\maketitle

\section{Prompt}
Translate the following C code to RISC-V assembly code. Use a minimum number of
instructions. Assume that the values of a, i, j, and r are in registers s1, s2, s3, and s4,
respectively. All the variables are signed. Load the constant into register s5 before the loop.
Write brief comments in your code. Clearly mark the instructions that controlling the outer
loop and the inner loop (for example, using different colors). There are 12 instructions in the
solutions.

\hfill
\begin{lstlisting}[language=C]
    for (i = 1; i < a; i++) 
        for (j = 0; j < i; j++) 
            r ^= (j + 0x55AABB33);
\end{lstlisting}

\section{Deliverables}
    \begin{lstlisting}
    # CSE 3666 Homework 1, Question 6

    .globl main
    .text
    
    main:   
        # use system call 5 to read integers
        addi    a7, x0, 5
        ecall
        addi    s1, a0, 0 # a in s1
    
        # initialize i, j, r to 0, and store them into s2, s3, s4 respectively
        addi    s2, x0, 0
        addi    s3, x0, 0
        addi    s4, x0, 0 
        
        # enter the loop routine
        j       loop_i
            
    loop_i:
        addi    s2, s2, 1          # i++
        blt     s2, s1, loop_j     # i < a, then run the inner loop
        bge     s2, s1, exit       # loop condition met, exit
        
    loop_j:
        addi    s3, s3, 1          # j++
        
        # load 20 most-significant-bits into the temporary register, t0,
        # and perform a sign-extension for the 12 remaining least
        # significant bits. with these two components known, add
        # them together and store the result in the temporary register, t1.
        # 
        # now that we have the full 0x55AABB33 value stored in a register,
        # we can add t1 + s3 to obtain the segment "j + 0x55AABB33" and
        # store it in t2. finally, compute the xor of the current value of r,
        # stored in register s4 against the result that is stored within t2.
        lui     t0, 0x55AAC
        addi    t1, t0, 0xFFFFFB33
        add     t2, s3, t1
        xor     s4, s4, t2
    
        blt     s3, s2, loop_j     # continue the loop until j >= i
            
    exit:
        # load service number 1 into the service register
        # and perform a system call to print the result
        # of the algorithm to the standard output.
        li, a7, 1      
        add a0, s4, x0 
        ecall
        
        # load service number 10 into the service register
        # and perform a system call to exit the program
        # with exit code 0.
        addi a7, x0, 10
        ecall
    \end{lstlisting}

\end{document}
