\documentclass{article}
\usepackage[utf8]{inputenc}

\title{CSE3666 — Lab 2}
\author{Mike Medved}
\date{January 29th, 2022}

\usepackage{color}
\usepackage{amsthm}
\usepackage{amssymb} 
\usepackage{amsmath}
\usepackage[margin=1in]{geometry} 
\usepackage{listings}
\usepackage{xcolor}
\usepackage{listings}

\lstset{frame=tb,
    language=[LaTeX]TeX,
    aboveskip=3mm,
    belowskip=3mm,
    showstringspaces=false,
    columns=flexible,
    basicstyle={\small\ttfamily},
    numbers=none,
    numberstyle=\tiny\color{gray},
    keywordstyle=\color{blue},
    commentstyle=\color{dkgreen},
    stringstyle=\color{mauve},
    breaklines=true,
    breakatwhitespace=true,
    tabsize=3
}

\lstdefinelanguage[mips]{Assembler}{%
  % so listings can detect directives and register names
  alsoletter={.\$},
  % strings, characters, and comments
  morestring=[b]",
  morestring=[b]',
  morecomment=[l]\#,
  % instructions
  morekeywords={[1]abs,abs.d,abs.s,add,add.d,add.s,addi,addiu,addu,%
    and,andi,b,bc1f,bc1t,beq,beqz,bge,bgeu,bgez,bgezal,bgt,bgtu,%
    bgtz,ble,bleu,blez,blt,bltu,bltz,bltzal,bne,bnez,break,c.eq.d,%
    c.eq.s,c.le.d,c.le.s,c.lt.d,c.lt.s,ceil.w.d,ceil.w.s,clo,clz,%
    cvt.d.s,cvt.d.w,cvt.s.d,cvt.s.w,cvt.w.d,cvt.w.s,div,div.d,div.s,%
    divu,ecall,eret,floor.w.d,floor.w.s,j,jal,jalr,jr,l.d,l.s,la,lb,lbu,%
    ld,ldc1,lh,lhu,li,ll,lui,lw,lwc1,lwl,lwr,madd,maddu,mfc0,mfc1,%
    mfc1.d,mfhi,mflo,mov.d,mov.s,move,movf,movf.d,movf.s,movn,movn.d,%
    movn.s,movt,movt.d,movt.s,movz,movz.d,movz.s,msub,msubu,mtc0,mtc1,%
    mtc1.d,mthi,mtlo,mul,mul.d,mul.s,mulo,mulou,mult,multu,mulu,mv,neg,%
    neg.d,neg.s,negu,nop,nor,not,or,ori,rem,remu,rol,ror,round.w.d,%
    round.w.s,s.d,s.s,sb,sc,sd,sdc1,seq,sge,sgeu,sgt,sgtu,sh,sle,%
    sleu,sll,sllv,slt,slti,sltiu,sltu,sne,sqrt.d,sqrt.s,sra,srav,srl,%
    srlv,sub,sub.d,sub.s,subi,subiu,subu,sw,swc1,swl,swr,syscall,teq,%
    teqi,tge,tgei,tgeiu,tgeu,tlt,tlti,tltiu,tltu,tne,tnei,trunc.w.d,%
    trunc.w.s,ulh,ulhu,ulw,ush,usw,xor,xori},
  % assembler directives
  morekeywords={[2].align,.ascii,.asciiz,.byte,.data,.double,.extern,%
    .float,.globl,.half,.kdata,.ktext,.set,.space,.text,.word},
  % register names
  morekeywords={[3]\$0,\$1,\$2,\$3,\$4,\$5,\$6,\$7,\$8,\$9,\$10,\$11,%
    \$12,\$13,\$14,\$15,\$16,\$17,\$18,\$19,\$20,\$21,\$22,\$23,\$24,%
    \$25,\$26,\$27,\$28,\$29,\$30,\$31,%
    \$zero,\$at,\$v0,\$v1,\$a0,\$a1,\$a2,\$a3,\$t0,\$t1,\$t2,\$t3,\$t4,
    \$t5,\$t6,\$t7,\$s0,\$s1,\$s2,\$s3,\$s4,\$s5,\$s6,\$s7,\$t8,\$t9,%
    \$k0,\$k1,\$gp,\$sp,\$fp,\$ra},
}[strings,comments,keywords]

\definecolor{CommentGreen}{rgb}{0,.6,0}
\lstset{
  language=[mips]Assembler,
  escapechar=@, % include LaTeX code between `@' characters
  keepspaces,   % needed to preserve spacing with lstinline
  basicstyle=\small\ttfamily\bfseries,
  commentstyle=\color{CommentGreen},
  stringstyle=\color{cyan},
  showstringspaces=false,
  keywordstyle=[1]\color{blue},    % instructions
  keywordstyle=[2]\color{magenta}, % directives
  keywordstyle=[3]\color{red},     % registers
}

\begin{document}

\maketitle

\section{Prompt}
In this lab, we write a program in RISC-V assembly language that prints 32 bits in a register.
$\hfill \break$
The program reads a signed integer from the console and prints the 32 bits in the integer, twice. 
$\hfill \break$
$\hfill \break$
Skeleton code in `lab2.s` already has the following steps. Study the code. 

\begin{enumerate}
    \item Read an integer using a system call and save it in `s1`.
    \item Use a system call to print `s1` in binary. 
\end{enumerate}

$\hfill \break$
Implement the following steps with RISC-V instructions.

\begin{enumerate}
    \item Use system call to print a newline character (ASCII value 10). Find the system call number in the document.
    Use system call to print a newline character (ASCII value 10). Find the system call
    number in the document.
    We can use ${\text{\\n}}$ as the immediate in instructions. 
    $\hfill \break$
    \item Write a loop to print the bits in `s1`, from bit 31 to bit 0. The printed bits should be the same as the ones printed by the system call.
    Think about the strategy/algorithm first. One method is provided in
    pseudocode at the bottom of this page.
    $\hfill \break$
    \item Use system call to print a newline character. 
\end{enumerate}

$\hfill \break$
Here are some example inputs/outputs of the program.

\begin{lstlisting}
-1
11111111111111111111111111111111
11111111111111111111111111111111

3666
00000000000000000000111001010010
00000000000000000000111001010010

20220201
00000001001101001000100100101001
00000001001101001000100100101001

-3666
11111111111111111111000110101110
11111111111111111111000110101110
\end{lstlisting}


\section{Deliverables}
    \begin{lstlisting}
    # CSE 3666 Lab 2

    .data
    newline: .string  "\n"
    
    .globl main
    .text
    
    main:          
        # use system call 5 to read integer
        addi    a7, x0, 5
        ecall
        addi    s1, a0, 0 # int in s1
    
        # use system call 35 to print a0 in binary
        # a0 has the integer we want to print
        addi    a7, x0, 35
        ecall
        
        # print newline
        la      a0, newline # load the address of the newline string
        li      a7, 4       # set the system call number to (PrintString)
        ecall               # system call
        
        addi    a7, x0, 1   # set system call number to 1 (PrintInt)
        addi    t0, t0, 32  # i = 32 (traverse bits backwards)
        addi    t1, t1, 1   # keep 1 in t1 for the mask in loop
        j loop              # enter loop routine
        
        loop:
            addi t0, t0, -1 # decrement loop counter
            
            # (k & (1 << n)) >> n
            sll  t2, t1, t0 # t2 = (1 << i)
            and  t3, s1, t2 # t3 = num & (1 << i)
            srl  a0, t3, t0 # a0 = (num & (1 << i)) >> i)
            ecall           # print extracted bit
            
            # restart loop, or exit if done
            bne t0, x0, loop
            beq t0, x0, loop_exit
            
        loop_exit:
                # print newline
            la a0, newline 
            li a7, 4
            ecall
    
            # exit program with exit code 0
            addi    a7, x0, 10      
            ecall
    \end{lstlisting}
    
\end{document}
