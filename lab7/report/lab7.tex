\documentclass{article}
\usepackage[utf8]{inputenc}

\title{CSE3666 — Lab 7}
\author{Mike Medved}
\date{March 30th, 2022}

\usepackage{color}
\usepackage{caption}
\usepackage{amsthm}
\usepackage{amssymb} 
\usepackage{amsmath}
\usepackage{listings}
\usepackage{listings}
\usepackage{graphicx}
\usepackage[margin=1in]{geometry} 
\usepackage[scaled=0.85]{FiraMono}
\usepackage[table]{xcolor}

\newcommand*\BitAnd{\mathbin{\&}}
\newcommand*\BitOr{\mathbin{|}}
\newcommand*\ShiftLeft{\ll}
\newcommand*\ShiftRight{\gg}
\newcommand*\BitNeg{\ensuremath{\mathord{\sim}}}

\lstset{basicstyle=\ttfamily, keywordstyle=\bfseries}
\lstset{
    language=Python,
    aboveskip=3mm,
    belowskip=3mm,
    showstringspaces=false,
    columns=flexible,
    basicstyle={\small\ttfamily},
    numbers=none,
    numberstyle=\tiny\color{gray},
    keywordstyle=\color{blue},
    commentstyle=\color{dkgreen},
    stringstyle=\color{mauve},
    breaklines=true,
    breakatwhitespace=true,
    tabsize=3
}

\definecolor{dkgreen}{rgb}{0,.6,0}
\definecolor{mauve}{rgb}{224,176,255}

\begin{document}
\graphicspath{ {.} }

\maketitle

\section{Prompt}
In this lab, we implement the multiplier we have discussed in lecture. The multiplicand and multiplier are of 8 bits (although we can easily change that). The multiplicand and the product are stored in 16-bit registers. The adder adds two 16-bit numbers.

$\hfill \break$
The internal signals needed in the implementation are already created in the skeleton code.

$\hfill \break$
The control module has been implemented. done is generated in the module. You may use done signal if needed. However, DO NOT use the counter register directly as the implementation may change.

The tasks to be completed in this lab are as follows. Budget 10 minutes for coding.

\begin{itemize}
    \item Instantiate registers for $p$, $x$, and $y$. Note that although $p$ is the output of the multiplier, it can be connected to the output of a register directly (i.e., driven directly by the register). $x$ and $y$ are internal signals in the multiplier.

    Let us call the instances reg\_p, reg\_x, and reg\_y, for $p$, $x$, and $y$, respectively.
    \item  Generate input signals to reg\_p in comb\_regs. p\_en is the only signal that needs to be set.    
\end{itemize}

\break
\section{Deliverables}
    \begin{lstlisting}[language=Python,frame=tb]
    ...

    @block
    def Mul2x(p, x_init, y_init, load, done, clock, reset):
    
    W = len(x_init)
    W2 = W + W

    adder_out = Signal(modbv(0)[W2:]) # output of adder
    x = Signal(modbv(0)[W2:]) 
    y = Signal(intbv(0)[W:])

    p_en = Signal(bool(1)) 
    x_en = Signal(bool(1))
    y_en = Signal(bool(1))

    counter = Signal(0)
    counter_in = Signal(0)
    counter_en = Signal(bool(1))
    reg_counter = RegisterE(counter, counter_in, counter_en, clock, reset) 

    p_reset = ResetSignal(bool(0), active=1, isasync=False)

    
    '''
    Step 1

    Instantiate registers for p, x, and y.
    Note that although p (product) is the output of the multiplier,
    and that it can be connected to the output of a register directly,
    while x and y are internal signals in the multiplier referring
    to the multiplier and multiplicand respectively.
    ''' 
    reg_p = RegisterE(p, adder_out, p_en, clock, p_reset)   
    reg_x = RegisterShiftLeft(x, x_init, load, x_en, clock, reset)
    reg_y = RegisterShiftRight(y, y_init, load, y_en, clock, reset)
    adder = Adder(adder_out, x, p)

    '''
    Step 2

    Generate input signals for p_en in comb_regs.
    p_en refers to the enable signal for the register p,
    and while true allows the register to receive a new input.
    '''
    @always_comb
    def comb_regs():
        p_reset.next = load
        p_en.next = y[0]
    
    ...
    \end{lstlisting}

\break
\section{Results}
The below tables demonstrate the step-by-step work done by the multiplier circuit we discussed in class. The following operations were performed on the circuit:
\begin{itemize}
    \item $17 \cdot 36 = 612$
    \item $36 \cdot 202 = 7272$
\end{itemize}

\begin{table}[!htb]
    \begin{tabular}{|c|c|
        >{\columncolor{green!20}}c |
        >{\columncolor{green!20}}c |
        >{\columncolor{green!20}}c |c|c|c|c|}
        \hline
        \textbf{load} & \textbf{cnt} & \textbf{prod} & \textbf{x} & \textbf{y} & \textbf{p\_en} & \textbf{x\_en} & \textbf{y\_en} & \textbf{done} \\ \hline
        1 & 0 & 0000000000000000 & 0000000000010001 & 00100100 & 0 & 1 & 1 & 0 \\ \hline
        0 & 1 & 0000000000000000 & 0000000000100010 & 00010010 & 0 & 1 & 1 & 0 \\ \hline
        0 & 2 & 0000000000000000 & 0000000001000100 & 00001001 & 1 & 1 & 1 & 0 \\ \hline
        0 & 3 & 0000000001000100 & 0000000010001000 & 00000100 & 0 & 1 & 1 & 0 \\ \hline
        0 & 4 & 0000000001000100 & 0000000100010000 & 00000010 & 0 & 1 & 1 & 0 \\ \hline
        0 & 5 & 0000000001000100 & 0000001000100000 & 00000001 & 1 & 1 & 1 & 0 \\ \hline
        0 & 6 & 0000001001100100 & 0000010001000000 & 00000000 & 0 & 1 & 1 & 0 \\ \hline
        0 & 7 & 0000001001100100 & 0000100010000000 & 00000000 & 0 & 1 & 1 & 0 \\ \hline
        0 & 8 & 0000001001100100 & 0001000100000000 & 00000000 & 0 & 1 & 1 & 1 \\ \hline
        0 & 8 & 0000001001100100 & 0010001000000000 & 00000000 & 0 & 1 & 1 & 1 \\ \hline
    \end{tabular}
\caption{17 $\cdot$ 36 $=$ 612}
\hfill \break
\begin{tabular}{|c|c|
    >{\columncolor{green!20}}c |
    >{\columncolor{green!20}}c |
    >{\columncolor{green!20}}c |c|c|c|c|}
    \hline
    \textbf{load} & \textbf{cnt} & \textbf{prod} & \textbf{x} & \textbf{y} & \textbf{p\_en} & \textbf{x\_en} & \textbf{y\_en} & \textbf{done} \\ \hline
    1 & 0 & 0000000000000000 & 0000000000100100 & 11001010 & 0 & 1 & 1 & 0 \\ \hline
    0 & 1 & 0000000000000000 & 0000000001001000 & 01100101 & 1 & 1 & 1 & 0 \\ \hline
    0 & 2 & 0000000001001000 & 0000000010010000 & 00110010 & 0 & 1 & 1 & 0 \\ \hline
    0 & 3 & 0000000001001000 & 0000000100100000 & 00011001 & 1 & 1 & 1 & 0 \\ \hline
    0 & 4 & 0000000101101000 & 0000001001000000 & 00001100 & 0 & 1 & 1 & 0 \\ \hline
    0 & 5 & 0000000101101000 & 0000010010000000 & 00000110 & 0 & 1 & 1 & 0 \\ \hline
    0 & 6 & 0000000101101000 & 0000100100000000 & 00000011 & 1 & 1 & 1 & 0 \\ \hline
    0 & 7 & 0000101001101000 & 0001001000000000 & 00000001 & 1 & 1 & 1 & 0 \\ \hline
    0 & 8 & 0001110001101000 & 0010010000000000 & 00000000 & 0 & 1 & 1 & 1 \\ \hline
    0 & 8 & 0001110001101000 & 0100100000000000 & 00000000 & 0 & 1 & 1 & 1 \\ \hline
\end{tabular}
\caption{36 $\cdot$ 202 $=$ 7272}
\end{table}

\end{document}
